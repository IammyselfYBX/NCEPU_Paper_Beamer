% 本章节是介绍项目的背景信息
\section{绪论}
\subsection{课题研究背景和意义}
在“中兴事件”刚过没多久,某国家就开始发起“华为事件”,进行各种制裁,对我国半导体产业的打击极大,放眼整个产业链,我们国家连一个像样的指令集架构都没有,当我们想研究开发产品竟然还需要别人来授权,这是莫大的耻辱。为什么别人能做出来,我们就做不出来,我们也并不比他们差什么。所以,我觉得我有必要研究这样的问题了,这就是我大学为什么在其他人都要研究人工智能这些内容的时候,我偏偏要研究这些偏底层的原理。

Risc-v是一个最近流行基于RISC指令集架构,类似与软件领域开源的linux一样,Risc-V也有很多领域可以发挥作用,如果将linux与RiSC-V结合一起来发展的话,我们国家可以利用已有的成就上完成自己的目标,而且开源的话可以大家一起参与进来,在这个过程中可以发现自己的不足,然后改正。


\subsection{国内外研究现状}
首先RISC-V是由国外的Berkeley大学开发的,我们承认与其之间存在差距,但是这都无所谓,就算他们现在比我们发展要靠前但这不代表永远都比我们要靠前,如果我们把精力,资源放在这方面,假以时日我们是会迎头赶上的,也可以做出属于自己的产业链。

目前国外SiFive公司做的不错,国内的芯来科技模仿SiFive做的也不错,阿里也有自己的研究。

\subsection{本文研究内容}
虽然,已经国内已经有人做出了基于RISC-V处理器内核,linus也把riscv-linux合并到Linux的主分支上了,但是我在这个毕设的内容就是自己进行一遍,是验证性的实验,虽然我也知道我做的这些东西微不足道,但是我想既然我开始做了,应该是会往好的方向发展的。

在过程中复习了一下本科所学得全部知识,这为我日后的发展打下了坚实的基础。


