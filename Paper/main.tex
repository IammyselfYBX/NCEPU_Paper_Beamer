% 华电毕业论文封面 
\documentclass[UTF8,a4paper]{ctexart}
\usepackage{geometry} % 设置页边距,页面大小
\geometry{left=2.5cm,right=2.0cm,top=2.50cm,bottom=2.0cm}
\linespread{1.3}%调整行间距
\usepackage{graphicx} % 引入图片
\pagestyle{plain}    
\usepackage{setspace} %行间距
\usepackage{array} %
\usepackage{booktabs} %调整表格线与上下内容的间隔
\usepackage{multirow}
\usepackage{hyperref} %加入书签跳转
\usepackage{fontspec} %引入字体Times New Roman字体
%\setmainfont{Times New Roman}             %设置正文字体为Times New Roman
\usepackage{abstract} % 修改摘要


%\usepackage[fontset=windows]{ctex}
\usepackage{xeCJK} %调用系统中已安装的字体
\setCJKmainfont{SimSun}
\setmainfont{Times New Roman}

%% 目录格式设置
\usepackage{tocloft}      %必须这么写,否则会报错
\renewcommand{\contentsname}{\centerline{\Large{\heiti{目\quad\quad 录}}}}
%\renewcommand{\cftchapleader}{\cftdotfill{0.6}} %设置chapter条目的引导点间距
\renewcommand{\cftsecleader}{\cftdotfill{0.6}}
\renewcommand{\cftsubsecleader}{\cftdotfill{0.6}}
\renewcommand{\cftsubsubsecleader}{\cftdotfill{0.6}}
%\renewcommand{\cftchapfont}{\hts}    %设置chapter条目的字体
\renewcommand{\cftsecfont}{\heiti}    %设置section条目的字体
\renewcommand{\cftsecfont}{\Large}    %设置section条目的四号
\renewcommand{\cftsubsecfont}{\songti} %设置subsection条目的字体
\renewcommand{\cftsubsecfont}{\large} %设置subsection条目的字体
\renewcommand{\cftsubsubsecfont}{\songti} %设置subsection条目的字体
\renewcommand{\cftsubsubsecfont}{\large} %设置subsection条目的字体

% 插入代码块
\usepackage{listings}
\usepackage{xcolor}
\lstset{
 %language=bash,                % the language of the code
 columns=fixed,       
% numbers=left,                                        % 在左侧显示行号
% numberstyle=\tiny\color{gray},                       % 设定行号格式
 frame=none,                                          % 不显示背景边框
 backgroundcolor=\color[RGB]{245,245,244},            % 设定背景颜色
 keywordstyle=\color[RGB]{40,40,255},                 % 设定关键字颜色
 numberstyle=\footnotesize\color{darkgray},           
 commentstyle=\it\color[RGB]{0,96,96},                % 设置代码注释的格式
% stringstyle=\rmfamily\slshape\color[RGB]{128,0,0},   % 设置字符串格式
}


% 参考文献
\usepackage{cite}
%%%%%%%%%%%%%%%%%%%%%%%%%%%%%%%%%%%%%%%%%%%%%%%%%%%%%%%%%%%%%%%%%%%%%%%%%%%%%%%%%
%  正文
\begin{document}
%% 封面
\input{part/cover}      
%%摘要
%%%% 中文摘要
% 这是中文摘要
\newpage
\pagenumbering{Roman}
\renewcommand{\abstractname}{\heiti{\Large{摘\quad \quad \quad \quad 要}}}
\begin{abstract}
{\large{不要少于400字该部分内容是放置摘要信息的。该部分内容是放置摘要信息的。该部分内容是放置摘要信息的。该部分内容是放置摘要信息的。该部分内容是放置摘要信息的。}}

%\\ \hspace*{\fill} \\  %换行,用空格填充,再换行,即可实现空出一整行的效果,不需任何环境调整
\noindent %取消首航缩进
    {\large{\textbf{关键字}:RISC-V,Qemu,linux}}
\end{abstract}

%%%% 英文摘要
% 这是英文摘要
\newpage
\renewcommand{\abstractname}{\Large \textbf{ABSTRACT}}
\begin{abstract}
{\large{This section is where the summary information is placed.This section is where the summary information is placed.This section is where the summary information is placed.}}

\noindent
\textbf{KEY WORDS:}RISC-V,Qemu,linux
\end{abstract}

%% 目录
\newpage
\tableofcontents
\thispagestyle{empty}

%% 章节
\newpage
\pagenumbering{arabic}
%%%%定制标题样式
%%%%% section
\CTEXsetup[name={第,章 },format={\centering\heiti\zihao{-2}},aftername={\enspace},beforeskip={24bp},afterskip={18bp}]{section} %name选项中不要使用中文逗号 \zihao(-2)字号小二
%%%%% subsection
\CTEXsetup[format={\raggedright\songti\zihao{-3}},aftername={\enspace},beforeskip={24bp},afterskip={6bp}]{subsection}
%%%%% subsubsection
\CTEXsetup[format={\raggedright\songti\zihao{4}},aftername={\enspace},beforeskip={12bp},afterskip={6bp}]{subsubsection}

% 本章节是介绍项目的背景信息
\section{背景}
在“中兴事件”刚过没多久,某国家就开始发起“华为事件”,对我国半导体产业的打击极大,放眼整个产业链,我们国家连一个像样的指令集架构都没有,我们想研究开发产品竟然还需要别人来授权,这是莫大的耻辱。
所以,我觉得是时候我应该研究一下了。
Risc-v是一个最近流行基于RISC指令集架构,类似与软件领域开源的linux一样,Risc-V也有很多领域可以发挥作用。

  % 介绍背景
% 本章节是介绍意义的
\section{意义}
虽然,已经国内已经有人做出了基于RISC-V处理器内核,linus也把riscv-linux合并到Linux的主分支上了,但是我在这个毕设的过程中复习了一下本科学得全部知识,这为我日后的发展打下了坚实的基础。


% 意义
% 本章节介绍RISC-V的原理
\section{Risc-V模拟器原理}
\subsection{KVM \& Qemu}
首先Qemu(Quick Emulator)本身并不完全是KVM的一部分,它是一套由软件模拟实现的。

而KVM(Kernel Virtual Machine)是有两部分组成,一部分是Linux内核的KVM模块,另一块是经过简化后的Qemu。它能够让Linux主机成为一个Hypervisor(虚拟机监控器)。在支持VMX(Virtual    Machine Extension)功能的x86处理器中,Linux在原有的用户模式和内核模式中新增加了客户模式,并且客户模式也拥有自己的内核模式和用户模式,虚拟机就是运行在客户模式中。三层结构如   图\ref{fig:kvm}

\begin{figure}[htbp]
  \centering %居中显示
  \includegraphics[width=0.6 \textwidth]{figs/KVM三种模式的层次关系.png}
  \caption{KVM三种模式的层次关系}
  \label{fig:kvm} %设置图形引用名称
\end{figure}
%『h』当前位置。将图形放置在正文文本中给出该图形环境的地方。如果本页所剩的页面不够,这一参数将不起作用。
%『t』顶部。将图形放置在页面的顶部。
%『b』底部。将图形放置在页面的底部。
%『p』浮动页。将图形放置在一只允许有浮动对象的页面上。
 %介绍RISC-V模拟器的原理
\input{part/process} % 介绍实验过程
% 本章节介绍与前人的比较
\section{与前人比较}
本实验以前有人在Github上已经实现了,并且发布成项目 \cite{BusyBear} ,但是在实验工程已经三年多没有更新,在根据README文档的时候发现其中很多功能已经失效了,很多步骤也会报错,但我依旧凭借着大学所学知识进行解决,并成功运行一个程序。

通过该实验,我对Qemu有更深的理解,再加上对操作系统底层有更深的理解,更深体会到编译工具的熟练使用。

虽然,本文已经可以运行一个程序,但是目前还有如下问题:(1)图形界面,就是用户需要使用图形界面的,这样可以更加方便用户使用,但是目前还没有想清楚如何添加图形界面方法;(2)需要做横向对比,和MIPS,ARM比较,看一下RISC-V的平台会提高多少性能;(3)其实这个也不算标准的移植,成功运行了,但是不代表日后会不会出现函数库,以及依赖的问题;

  %与前人比较


根据 David A. Patterson 的著作\cite{2006Computer}

%% 参考文献
\bibliographystyle{plain}
\bibliography{Ref}
%\bibliographystyle{plain}指定参考文献的呈现方式,常见的预设样式的可选项有8种,分别是:
%% 1. plain,按字母的顺序排列,比较次序为作者、年度和标题;
%% 2. unsrt,样式同plain,只是按照引用的先后排序;
%% 3. alpha,用作者名首字母+年份后两位作标号,以字母顺序排序;
%% 4. abbrv,类似plain,将月份全拼改为缩写,更显紧凑;
%% 5. ieeetr,国际电气电子工程师协会期刊样式;
%% 6. acm,美国计算机学会期刊样式;
%% 7. siam,美国工业和应用数学学会期刊样式;
%% 8. apalike,美国心理学学会期刊样式;
\end{document}
